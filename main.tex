\documentclass{article}
\usepackage{fancyvrb}
%--- listing
\usepackage{color}
\usepackage{listings}
\usepackage[a4paper, total={6in, 8in}]{geometry}
\usepackage{polski}
\usepackage[utf8]{inputenc}
\usepackage{placeins}


%\usepackage{courier} %caused problems for me

\lstloadlanguages{% Check Documentation for further languages ...
C,
C++,
csh,
Java
}

\definecolor{red}{rgb}{0.6,0,0} % for strings
\definecolor{blue}{rgb}{0,0,0.6}
\definecolor{green}{rgb}{0,0.8,0}
\definecolor{darkgreen}{rgb}{0,0.2,0}
\definecolor{cyan}{rgb}{0.0,0.6,0.6}

\lstset{
language=csh,
basicstyle=\footnotesize\ttfamily, 
numbers=left, 
numberstyle=\tiny, 
numbersep=5pt, 
tabsize=2, 
extendedchars=true, 
breaklines=true, 
frame=b,
stringstyle=\color{blue}\ttfamily, 
showspaces=false, 
showtabs=true, 
xleftmargin=17pt,
framexleftmargin=17pt,
framexrightmargin=5pt,
framexbottommargin=4pt,
commentstyle=\color{darkgreen},
morecomment=[l]{//}, %use comment-line-style!
morecomment=[s]{/*}{*/}, %for multiline comments
showstringspaces=false, 
morekeywords={  abstract, event, new, struct, Auth, AccessType,AuthContext, AuthFilter, AclTables, T, RoleTree, RoleNode,
                as, explicit, null, switch,
                base, extern, object, this,
                bool, false, operator, throw,
                break, finally, out, true,
                byte, fixed, override, try,
                case, float, params, typeof,
                catch, for, private, uint,
                char, foreach, protected, ulong,
                checked, goto, public, unchecked,
                class, if, readonly, unsafe,
                const, implicit, ref, ushort,
                continue, in, return, using,
                decimal, int, sbyte, virtual,
                default, interface, sealed, volatile,
                delegate, internal, short, void,
                do, is, sizeof, while,
                double, lock, stackalloc,
                else, long, static,
                enum, namespace, string},
keywordstyle=\color{cyan},
identifierstyle=\color{red},
 literate={ą}{{\k{a}}}1
             {Ą}{{\k{A}}}1
             {ę}{{\k{e}}}1
             {Ę}{{\k{E}}}1
             {ó}{{\'o}}1
             {Ó}{{\'O}}1
             {ś}{{\'s}}1
             {Ś}{{\'S}}1
             {ł}{{\l{}}}1
             {Ł}{{\L{}}}1
             {ż}{{\.z}}1
             {Ż}{{\.Z}}1
             {ź}{{\'z}}1
             {Ź}{{\'Z}}1
             {ć}{{\'c}}1
             {Ć}{{\'C}}1
             {ń}{{\'n}}1
             {Ń}{{\'N}}1
}
\usepackage{caption}
\DeclareCaptionFont{white}{\color{black}}
\DeclareCaptionFormat{listing}{\colorbox{white}{\parbox{\textwidth}{\hspace{15pt}#1#2#3}}} %changed \colorbox{8} to \colorbox{blue} cause 8 is not a color!
\captionsetup[lstlisting]{format=listing,labelfont=white,textfont=white, singlelinecheck=false, margin=0pt, font={bf,footnotesize}}

%--- listing end
\title{Inteligentny system wykrywania zagrożeń w górach} % Title of the assignment

\author{Jakub Natonek, Nikita Grygoriev} % Author name and email address
\usepackage{tikz}
\usepackage{enumerate}

% \documentclass[tikz,border=2mm]{standalone}
 \usetikzlibrary{matrix,positioning}

\begin{document}

\maketitle 

\tableofcontents
\newpage

\section{Streszczenie systemu} 
Niniejsze opracowanie modeluje działanie systemu ostrzegawczego przed zagrożeniami występującymi w górach.
Celem pracy jest zapewnienie bezpieczeństwa turystów oraz optymalizacja pracy ratowników GOPR. Rozwiązanie jest oparte na danych pobranych z rozmieszczonych czujników BTS, dronów BTS oraz dronów z kamerami które lokalizują smartphone'y turystów na danym terenie, co umożliwia ich monitorowanie oraz wysyłanie ostrzeżeń o pojawiających się zagrożeniach.

\section{Lista obiektów}
\begin{itemize}
  \item Aplikacja mobilna
  \item Komputer ratownika
  \item Zwierzęta
  \item Stacje / sensory pogodowe
  \item Drony (BTS / z kamerą)
  \item Stacje BTS
  \item Blockchain

  
\end{itemize}
\section{Lista bodźców zewnętrznych}
\begin{itemize}
  \item Czujniki BTS: wykrywanie nowego smartphone'a z aplikacją mobilną w rejonie czujnika 
  \item Smartphone: wysyłka lokalizacji turysty
  \item Sensory pogodowe: wykrywanie niebezpiecznych zjawisk atmosferycznych
  \item Drony BTS: zwiększanie dokładności lokalizacji obiektów
  \item Drony z kamerą: bezpośrednie obserwowanie pozycji
  \item Blockchain: przechowywanie wykrytych zagrożeń lokalizacji turystów
\end{itemize}
\clearpage

\section{Diagramy kontekstowe} 


\subsection{Proponowany diagram oddziaływań obiektu z kontekstem pogodowym -- W i nie-pogodowym -- S}
\begin{figure}[!htb]
\centering
\includegraphics[scale=0.5]{Diagrams-Obiekt.png}
\caption{E6g -- odległość od przewodniczego, E6a -- odległość od niebezpiecznych zwierząt, E6m -- ruch / brak ruchu, E6r -- znajdywanie się na szlaku / poza szlakiem \hspace{\textwidth} E1-5: Charakter zjawisk pogodowych: lawina, wiatr, mgła, temperatura, deszcz / burza oraz parametry czasowe: pora dnia, pora roku}
\end{figure}
\FloatBarrier
\clearpage
\subsection{Podział obiektów na źródła, wydarzenia i odbiorców (z wykorzystaniem blockchain)}
\begin{figure}[!htb]
\centering
\includegraphics[scale=0.6]{Diagrams-Źródło i odbiorca.png}
\caption{Diagram opisujący rolę blockchain w systemie -- użycie danej bazy danych zapewnia spójność informacji dotyczących pozycji turystów i zwierząt oraz wykrytych zagrożeń}
\end{figure}
\FloatBarrier


\subsection{Opracowanie algorytmu}
\begin{figure}[!htb]
\centering
\includegraphics[scale=0.5]{Diagrams-Algorytm.png}
\caption{Diagram opisujący algorytm pracy z danymi}
\end{figure}
\FloatBarrier
\clearpage

\subsection{Uogólniony diagram oddziaływań w układzie}
\begin{figure}[!htb]
\centering
\includegraphics[scale=0.35]{Diagrams-System.png}
\caption{Diagram opisujący hierarchię akcji i obiektów. Blockchain jest stosowany do przechowywania pozycji zwierząt i turystów oraz zarejestrowanych zagrożeń}
\end{figure}
\FloatBarrier

\subsection{Szczegółowy diagram oddziaływań w układzie}
\begin{figure}[!htb]
\centering
\includegraphics[scale=0.35]{Diagrams-System (szczegółowy).png}
\caption{Diagram łączący algorytm pokazany na rysunku 4.3 oraz uogólniony diagram oddziaływań pokazany na rysunku 4.4}
\end{figure}
\FloatBarrier

\subsection{Struktura blockchain}
\begin{figure}[!htb]
\centering
\includegraphics[scale=0.35]{Diagrams-Blockchain.png}
\caption{Blockchain zawiera w sobie 3 bazy danych pokazane na rysunku}
\end{figure}
\FloatBarrier
\clearpage
\section{Diagramy DFD} 

\subsection{DFD Poziom 0}
\begin{figure}[!htb]
\centering
\includegraphics[scale=0.5]{Diagrams-POZIOM 0.png}
\caption{Ogólny opis działania systemu}
\end{figure}
\FloatBarrier

\subsection{Poziom 1.1}
\begin{figure}[!htb]
\centering
\includegraphics[scale=0.6]{Diagrams-1.1.png}
\caption{Identyfikacja nowego smartphone'u pojawiającego się na terenie monitorowanym}
\end{figure}
\FloatBarrier
\clearpage

\subsection{Poziom 1.2}
\begin{figure}[!htb]
\centering
\includegraphics[scale=0.5]{Diagrams-1.2.png}
\caption{Integracja układu z blockchain}
\end{figure}

\subsection{Poziom 1.3}
\begin{figure}[!htb]
\centering
\includegraphics[scale=0.5]{Diagrams-1.3.png}
\caption{Analiza poziomu ryzyka}
\end{figure}
\FloatBarrier
\clearpage

\subsection{Poziom 1.4}
\begin{figure}[!htb]
\centering
\includegraphics[scale=0.5]{Diagrams-1.4.png}
\caption{Wybór reakcji zależnie od poziomu wykrytego ryzyka}
\end{figure}
\FloatBarrier
\clearpage
\end{document}
